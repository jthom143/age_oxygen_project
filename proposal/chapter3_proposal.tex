%++++++++++++++++++++++++++++++++++++++++
% Don't modify this section unless you know what you're doing!
\documentclass[article,12pt]{article}
\usepackage{tabularx} % extra features for tabular environment
\usepackage{amsmath}  % improve math presentation
\usepackage{graphicx} % takes care of graphic including machinery
\usepackage[margin=1in,letterpaper]{geometry} % decreases margins
\usepackage{cite} % takes care of citations
\usepackage{sectsty} % change font size of section headers
\sectionfont{\fontsize{12}{15}\selectfont}

%++++++++++++++++++++++++++++++++++++++++


\begin{document}

\section{Chapter 3 Plan:}
\begin{enumerate}
  \item What is the relationship between age and oxygen in GFDL ESM2Mc?
  \begin{enumerate}
      \item What is the spatial variability of this relationship?
      \item What does this relationship say about the underlying mechanisms changing
      oxygen concentration?
  \end{enumerate}
  \item How have Southern Ocean ventilation rates have changed over the last three
  decades using ship-based repeat hydrography measurements of CFC-11 and CFC-12
  (following method used in Waugh et al., 2013).
  \item What is the impact of changing circulation on oxygen?
\end{enumerate}

\section{Motivation}
\begin{itemize}
  \item Recent data assimilating ocean circulation model study suggest strength of upper-ocean overturning circulation has
  strengthened in the last decade (2000--2009) resulting in an increase of the upper
  ocean carbon sink~\cite{Devries2017}.
  \item This is a direct reversal in the trend documented by observations and modeling
  studies in the previous decade (1990--1999)~\cite{Lovenduski2007,Waugh2013b,LeQuere2007e}.
  \item Unclear whether these decadal trends are a result of changing anthropogenic
  forcing, or decadal variability in ocean circulation.
  \item Also observed decrease in oxygen concentration in the Southern Ocean over
  the period 1960--2009 is consistent with a previous study citing an observed decline
  of deep water formation of Antarctic water masses~\cite{Purkey2012} and may also represent
  changes in the wind-driven ventilation~\cite{Schmidtko2017}.
\end{itemize}

\section{Model Analysis:}
\noindent{\textbf{What is the relationship between age and oxygen in GFDL ESM2Mc?}}
\begin{itemize}
    \item What is the spatial variability of this relationship?
    \item What does this relationship say about the underlying mechanisms changing
    oxygen concentration?
\end{itemize}

\noindent{\textbf{To Do List:}}
\begin{enumerate}
  \item Plot correlation and regression coefficients between age and oxygen for
  each depth level in ocean.
  \item Define and calculate AOU and OUR.
  \item Plot zonal averaged age, oxygen, and AOU for each ocean basin (Atlantic,
  Pacific, Indian)
  \item Plot correlation between age and oxygen for lat vs depth for each ocean basin.
\end{enumerate}


\noindent{\textbf{Ventilation, overturning and consumption.}}
Current understanding of oceans attributes deep-ocean oxygen loss to four processes:
\begin{enumerate}
  \item A reduction of ventilation in deep convective regions which provides less
  ventilated waters in high latitudes and a reduction of mixed layer subduction in
  mid and high latitudes. The timescales of these changes are long (50-100 years) because
  of the time it takes for oxygen-reduced waters to advect through the global basins
  before affecting older waters.
  \item A slowdown of the meridional overturning circulation which would reduce the amount
  of oxygenated waters that are mixed with older waters and thus increase the age
  of deep waters in general.
  \item An increase in biological activity in the upper ocean. This would increase
  remineralization and thus oxygen consumption at depth.
  \item Natural multi-decadal variability that is not accurately captured by the
  sparse data available.
\end{enumerate}

\noindent{\textbf{TTD Analysis to Constrain Estimates of Age.}}

Why use transit time distributions?
\begin{itemize}
  \item Tracer ages from different tracers yield different times. It is not clear
  what aspects of the transport are measured by different tracers.
  \item Can use a \textit{distribution of transit times} to compare the timescales
  derived from different tracers.
  \item It is the distribution, rather than the particular tracers age, that is a
  fundamental descriptor of the flow.
  \item Each tracer, because of its different boundary condition or decay rate,
  weights the features of the distribution differently.

\end{itemize}

\noindent{\textbf{Oxygen Distribution in the Oceans}}

\noindent{\textit{Separation of Preformed and Remineralized Components:}}

To study the impact of aerobic remineralization on the ocean interior distribution
of oxygen we can estimate the remineralized component i.e. the change in O$_2$
since the water parcel was last in contact with the atmosphere:
$$
\Delta[O_2]_{remin} = [O_2]_{observed} + [O_2]_{preformed}
$$

For oxygen, the surface ocean is generally close to saturation (the gas exchange of
oxygen is more rapid than the residence time at the surface). Therefore we can
generally assume that the $[O_2]_{preformed}=[O_2]_{saturation}$. Since the saturation
can be computed from the potential temperature and salinity, we can thus estimate
the remineralized component of oxygen, for which the term \textit{Apparent Oxygen
Utilization} is commonly used:
$$
AOU = [O_2]_{sat} - [O_2]
$$
 where $AOU = - \Delta[O_2]_{remin}$. The assumption that the preformed oxygen concentration
 is equal to the oxygen saturation is not always valid. In particular, the low latitude
 ocean is slightly supersaturated in oxygen. These supersaturated waters ventilate 





\bibliographystyle{abbrv}
\bibliography{/RESEARCH/library.bib}
\end{document}
